\documentclass[english]{article}
\usepackage{amsmath}

\makeatletter

\makeatother

\usepackage{babel}
\setcounter{secnumdepth}{5}
\begin{document}

\title{List of uncompleted projects}


\author{Eric Moyer}


\date{30 August 2012}

\maketitle

\section{Raw list of projects}

\begin{enumerate}
\item Hough peak matching (April - June 2011)
      \begin{enumerate}
      \item Explored other ideas for peak-matching using SVD
      \item Implemented own version of Hough peak matching (though it still needs 
            an automatic bump hunter - I was using PRIM but got interrupted). 
            PRIM is a good idea to experiment with because the most practical of
            my analytical attempts uses it too. June has the PRIM experiments - 
            I had to combine it with thresholding.
      \item Exploreed possibility of analytical Hough peak finding 
            (See May lab book entries)
      \end{enumerate}
\item Outlier detection (June - July 2011) - which methods work best in NMR
      metabolomics context
      \begin{enumerate}
      \item I did extensive background reading
      \item I compiled a list of methods to compare - most of which are available
            as different types of source package
      \end{enumerate}
\item Targeted deconvolution (July - December 2011)
      \begin{enumerate}
      \item Implemented nice UI for targeted deconvolution using curve fitting
      \end{enumerate}
\item Bite Diet (August 2011 - Present)
      \begin{enumerate}
      \item Have kludgy manual system that works for me
      \item Will write a less kludgy manual system for my wife
      \item Attended patent seminar. Still need to talk to WSU Randy Raider.
      \item Could start on signal processing automated version any day (though
            I'd like to use swallows rather than bites. I have been too lazy to
            start the swallows manually.
      \end{enumerate}
\item Machine Learning Peak Detection (September 2011 - )
      \begin{enumerate}
      \item Outgrowth of bad peak detection in both Hough project and
            targeted deconvolution
      \item Involved modification of waffles to 
      \item Lots of normal classifiers didn't work well due to the problem
            structure (needs a max operation) and the very unbalanced input
            distribution.
      \item Wrote but never tested an approximation to the Bayesian problem
      \item Involved writing a Linear assignment solver for a loss
            function. Probably a better way of computing the error than treating
            the list of "this is a peak, this isn't" as a vector. Something I
            didn't realize at the time, but the degree of freedom this introduces
            due to its optimistic nature --- error is calculated on the 
            best assignment of predicted to actual --- will prevent comparing
            different numbers of peaks since fewer will be able to get a lower
            score due to the extra, optimistic degree of freedom.
      \end{enumerate}
\item HMDB Cleanup (September 2011)
\item MIC
\end{enumerate}


\end{document}
