\documentclass[english]{article}
\usepackage{amsmath}

\makeatletter

\makeatother

\usepackage{babel}

\usepackage[english = american]{csquotes}
\MakeOuterQuote{"}

\usepackage{color}

\setcounter{secnumdepth}{5}

\usepackage[round,sort]{natbib}

%
% My Commands 
%
\newcommand{\todo}[1]{\textcolor{red}{\textbf{TODO:} #1}}


\begin{document}

\title{Dilution normalization methods in \textsuperscript{1}H NMR metabolomics data}

\author{Eric Moyer, Michael Raymer}

\date{1 November 2012}

\maketitle

\begin{abstract}
This is where the abstract will eventually go. Lorem ipsum dolor sit amet, consectetur adipiscing elit. Curabitur vel metus nulla, ac auctor elit. Cras risus quam, scelerisque nec eleifend a, pellentesque in lacus. Duis justo metus, egestas non congue in, tincidunt et justo. Class aptent taciti sociosqu ad litora torquent per conubia nostra, per inceptos himenaeos. Curabitur porttitor venenatis lacus, id tristique nisi commodo in. Aliquam porttitor aliquet arcu, eu eleifend erat pretium vel. Duis non ultricies lorem. Nam eu nisi eget nisl viverra congue a ac ligula.
\end{abstract}

\section{Introduction}
Metabolomics measures the small molecules used as inputs, substrates, or outputs of metabolic processes as a means of inferring the nature of, or changes to those processes. In any particular experiment, the experimenters are only interested in processes connected with a phenomenon under study. They design the experiment to manipulate a few variables related to that phenomenon. The change in measured variables that is a response to the experimental manipulation is called induced variation. All other changes are called uninduced variation.

Uninduced variation can be broken down into two classes. The first is technical variation: the changes introduced in the measurements by the process of measurement. The second class is uninduced biological variation and covers those changes due to the complexity of the biological system's interaction with uncontrollable factors.

Because uninduced variation confounds the measurement of the induced variation, all experimentalists strive to design their experiments in a way that will minimize uninduced variation or at least make it distinguishable from the induced variation. Animals are placed in random cages, standard diets are adhered to, and temperature is carefully regulated to minimize uninduced variation. However, some is unremovable. This is attenuated through a set of methods known as normalization.

Different normalization methods try to deal with different problems. For example, auto-scaling (also called unit-variance or uv scaling) removes variation in absolute concentration between metabolites\cite{VanDenBerg2006}. This paper will focus on normalization methods used to deal with sample dilution in NMR metabolomics spectra.

Experimental subjects frequently excrete or store different amounts of water for reasons that have nothing to do with the factor under study. Since NMR is mainly used to measure differences in concentration of various metabolites, water dilution results in a global change in the apparent concentrations of other elements measured. This change is well approximated by multiplying the measured amplitudes in a frequency-domain spectrum by a dilution-dependent constant.\footnote{\citep{Zhang2009a} mentions that some of the smaller peaks had a non-linear response to dilution. However, we have seen no follow-up on this observation and feel that it might be explainable by technical variation. For example they measured peak heights could have been deceived by peak widths changing.} Dilution normalization techniques are those which attempt to recover this multiplicative constant by using statistical techniques. Technical limitations of NMR spectroscopy usually require the elimination of any water signal during spectrum acquisition, so the constant recovery must be done by comparing the statistical characteristics of the input spectra.

\todo{At the end, check whether max-binning PQN would have been adversely affected by peak width being dependent on dilution factor.}

\section{Related Work}
Because of the importance of this problem, many dilution normalization methods have been proposed. However, before now, there has not been a comprehensive comparison of all of them under a wide variety of conditions.

\citep{VanDenBerg2006} compares their probabilistic quotient normalization (PQN) with constant sum normalization (CSN) on some highly simplified simulated datasets and on urine samples where one group had extreme glucose excretion. PQN outperformed CSN. At the time, they said that "no investigations concerning the normalization step of a set of NMR spectra of complex biofluids have been published." They recommended that small samples normalize to a reference spectrum generated only from the control population.

\citep{Torgrip2008} followed up with another algorithm, histogram matching normalization (HMN). They tested PQN, HMN and CSN on simulated \textit{Arabidopsis thaliana} tissue extracts digitally manipulated before analysis introducing a simulated dilution and simulated highly variable large peaks. They also looked at the performance of their algorithm on mouse urine. HMN outperformed its competitors on the simulated data, but mouse urine results were inconclusive. \todo{Does the ethionine treated mouse-urine lack a large highly variable peak?}.

Neither of these papers used data without a large, highly variable peak, which is a well-known weakness of CSN \citep{VanDenBerg2006}.

\citep{Wen2007} compared 1-Norm, which is CSN but with an absolute value inside the summation with 2-Norm, for datasets that compare diabetes with normal populations. They concluded that both methods produced adequate results but 2-Norm gave better separation in PCA space.

\citep{Schnackenberg2007} used two normalization factors based on the weight of the subject as well as creatinine normalization (CRN) and CSN. They noted that correlations between the normalization factor and age could lead to additional PCA separation between age groups and that of the factors studied CSN was least correlated to age.

\citep{Zhang2009a} used real spectra of spike-in urine samples with 3 different dilution levels. Then they compared CRN, CSN, and variance stabilizing normalization (VSN.) They found that VSN did the best in removing dilution artifacts in the spike-in metabolites and the PCA plot but that CSN was superior when using a T-test to confirm differences.

\citep{DeMeyer2010a} made a detailed comparison of 14 CSN and PQN variants on plasma samples. They compared NMR metabolite concentration proxies with independent measurements of 3 metabolites of different peak heights. Their conclusion was that CSN was inferior and that PQN after adaptive-intelligent binning and maximum variable definition was the method best for general use.

\citep{Kohl2012} compared 10 normalization methods but were focused on dealing with non-dilution uninduced variation. They kept dilution out of their spike-in dataset through careful measurement and they pre-normalized their urine spectra with CRN to remove dilution effects. Still, PQN had the best between-sample normalization while retaining the most "biological" data from the spike-in metabolites. Other methods came to the fore when considering other factors like classification performance on an SVM or removing heteroscedasticity. Quantile Normalization and Cubic spline were their ultimate recommendations. They calculated the expected fold-change versus actual fold change, but only plotted it for 3 of the 10 normalization methods.

\section{Dilution Recovery Normalization Algorithms}
\section{Methods}

\bibliographystyle{nourlplainnat}
\bibliography{/home/eric/Papers/library}
\end{document}
