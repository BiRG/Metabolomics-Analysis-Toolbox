\documentclass[english]{article}
\usepackage{amsmath}

\makeatletter

\makeatother

\usepackage{babel}
\usepackage{color}
\setcounter{secnumdepth}{5}

%
% My Commands 
%
\newcommand{\todo}[1]{\textcolor{red}{\textbf{TODO:} #1}}


\begin{document}

\title{Dilution normalization methods in \textsuperscript{1}H NMR metabolomics data}

\author{Eric Moyer, Michael Raymer}

\date{1 November 2012}

\maketitle

\begin{abstract}
This is where the abstract will eventually go. Lorem ipsum dolor sit amet, consectetur adipiscing elit. Curabitur vel metus nulla, ac auctor elit. Cras risus quam, scelerisque nec eleifend a, pellentesque in lacus. Duis justo metus, egestas non congue in, tincidunt et justo. Class aptent taciti sociosqu ad litora torquent per conubia nostra, per inceptos himenaeos. Curabitur porttitor venenatis lacus, id tristique nisi commodo in. Aliquam porttitor aliquet arcu, eu eleifend erat pretium vel. Duis non ultricies lorem. Nam eu nisi eget nisl viverra congue a ac ligula.
\end{abstract}

\section{Introduction}
Metabolomics investigates the small molecules used as inputs, substrates, or outputs of metabolic processes as a means of inferring the nature of, or changes to those processes. In any particular experiment, the experimenter is only interested in a few processes, namely those connected with the phenomenon under study. Metabolic changes due to other phenomena are experimental noise that can make it impossible to determine the effects of the desired phenomenon. Thus scientists attempt to remove extraneous changes so they can focus on the ones that provide them the information they need to answer the questions they are investigating.

Many noise factors are removed by careful experimental design and planning. Animals are placed in random cages, standard diets are adhered to, and temperature is carefully regulated to minimize differences between groups not caused by the current treatment. However, not all factors can be removed in this way. Those that cannot are removed or attenuated through a set of methods known as normalization.

In NMR metabolomics, one common noise factor is the dilution of the sample. Dilution effects are very common in urine spectra, but they also appear in other biofluids. Experimental subjects frequently excrete or store different amounts of water for reasons that have nothing to do with the factor under study. Since NMR is mainly used to measure differences in concentration of various metabolites, water dilution results in a global change in the apparent concentrations of other elements measured. This change is well approximated by multiplying the measured amplitudes in a frequency-domain spectrum by a dilution-dependent constant.\footnote{(Zhang 2009) mentions that some of the smaller peaks had a non-linear response to dilution. However, we have seen no follow-up on this observation indicating a better model for this response, thus we will continue with the traditional approximation.} Dilution normalization techniques are those which attempt to recover this multiplicative constant by using statistical techniques. Technical limitations of NMR spectroscopy usually require the elimination of any water signal during spectrum acquisition, so the constant recovery must be done by comparing the statistical characteristics of the input spectra.

\section{Related Work}
Because of the importance of this problem, many dilution normalization methods have been proposed. However, before now, there has not been a comprehensive comparison of all of them under a wide variety of conditions.

(Dieterle 2006) compares their probabilistic quotient normalization (PQN) with constant sum normalization (CSN) on some highly simplified simulated datasets and on urine samples where one group had extreme glucose excretion. PQN outperformed CSN. At the time, they said that "no investigations concerning the normalization step of a set of NMR spectra of complex biofluids have been published." They recommended that small samples normalize to a reference spectrum generated only from the control population.

(Torgrip 2007) followed up with another algorithm, histogram matching normalization (HMN). They tested PQN, HMN and CSN on simulated \textit{Arabidopsis thaliana} tissue extracts digitally manipulated before analysis introducing a simulated dilution and simulated highly variable large peaks. They also looked at the performance of their algorithm on mouse urine. HMN outperformed its competitors on the simulated data, but mouse urine results were inconclusive. \todo{Does the ethionine treated mouse-urine lack a large highly variable peak?}.

Neither of these papers used data without a large, highly variable peak, which is the Achilles heel of CSN.

(Schnackenberg 2007) \todo{Need to read this paper, then rewrite}

\section{Dilution Recovery Normalization Algorithms}
\section{Methods}
\end{document}
